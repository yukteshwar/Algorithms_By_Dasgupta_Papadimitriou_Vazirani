\documentclass[11pt]{article} 

\usepackage[utf8]{inputenc} 
\usepackage{listings}

%%% PAGE DIMENSIONS
\usepackage{geometry} 
\geometry{letterpaper} % or 
%\geometry{a4paper} % or letterpaper (US) or a5paper or....
% \geometry{margin=2in} % for example, change the margins to 2 inches all round
% \geometry{landscape} % set up the page for landscape
%   read geometry.pdf for detailed page layout information

\usepackage{graphicx} % support the \includegraphics command and options

% \usepackage[parfill]{parskip} % Activate to begin paragraphs with an empty line rather than an indent

%%% PACKAGES
\usepackage{booktabs} % for much better looking tables
\usepackage{array} % for better arrays (eg matrices) in maths
\usepackage{paralist} % very flexible & customisable lists (eg. enumerate/itemize, etc.)
\usepackage{verbatim} % adds environment for commenting out blocks of text & for better verbatim
\usepackage{subfig} % make it possible to include more than one captioned figure/table in a single float
% These packages are all incorporated in the memoir class to one degree or another...
\usepackage{amsmath} %for mathematical equations
%%% HEADERS & FOOTERS
\usepackage{fancyhdr} % This should be set AFTER setting up the page geometry
\usepackage{enumitem}
\pagestyle{fancy} % options: empty , plain , fancy
\renewcommand{\headrulewidth}{0pt} % customise the layout...
\lhead{}\chead{}\rhead{}
\lfoot{}\cfoot{\thepage}\rfoot{}

%%% SECTION TITLE APPEARANCE
\usepackage{sectsty}
\allsectionsfont{\sffamily\mdseries\upshape} % (See the fntguide.pdf for font help)
% (This matches ConTeXt defaults)

%%% ToC (table of contents) APPEARANCE
\usepackage[nottoc,notlof,notlot]{tocbibind} % Put the bibliography in the ToC
\usepackage[titles,subfigure]{tocloft} % Alter the style of the Table of Contents
\renewcommand{\cftsecfont}{\rmfamily\mdseries\upshape}
\renewcommand{\cftsecpagefont}{\rmfamily\mdseries\upshape} % No bold!

%%% END Article customizations

%%% The "real" document content comes below...

\title{Algorithms\\By\\Sanjoy Dasgupta Christos Papadimitriou Umesh Vazirani \\ Chapter 1: Algorithms with numbers}
\author{Yukteshwar Baranwal}
%\date{} % Activate to display a given date or no date (if empty),
% otherwise the current date is printed 
\newcommand{\dd}[1]{\mathrm{d}#1}

\begin{document}
	\maketitle
	
	\textbf{Problem 1.1:} Show that in any base $b \geq 2$, the sum of any three single-digit numbers is at
	most two digits long.
	
	Maximum possible number in base $b$ is $b-1$ e.g. 1 in binary($b = 2$) and 9 in decimal($b=10$). On adding $b-1$ thrice, the result will be $3\times (b-1)$, which is at most 2 bit long as $3\times (b-1) < b^3$.\\
	
	\textbf{Problem 1.2:} Show that any binary integer is at most four times as long as the corresponding decimal integer. For very large numbers, what is the ratio of these two lengths,	approximately?
	
	A number of $n$ in base $b$ has length $log(n)/log(b)$. So , length for binary is $log(n)/log(2)$ \& for decimal is $log(n)/log(10)$. Their ratio is given by $log(10)/log(2) = 1/0.3010 > 3$ and its ceil value is 4.\\
	
	\textbf{Problem 1.3:} A d-ary tree is a rooted tree in which each node has at most d children. Show that any d-ary tree with n nodes must have a depth of $\Omega(log(n)/log(d))$. Can you give a precise formula for the minimum depth it could possibly have?
	
	For n child nodes and for a complete tree, minimum depth would be $log(n)/log(d)$. The sum of nodes over levels is a geometric series sum having value $(d^{k+1}-1)/(d-1)$ where k is depth of tree and this sum would be n. Hence, k would be $\Omega(log(n)/log(d))$. Precise representation would be $\lfloor log_d(n)\rfloor$.\\
	
	\textbf{Problem 1.4:} Show that $log(n!) = \Theta(nlog(n))$.
	
	Using hint, $n! < n^n$ \& $n! > (n/2)^{n/2}$. Hence, $log(n!) < nlog(n)$ \& $log(n!) < (n/2)log(n/2)$. The lower bound can also be expressed as $log(n!) < (1/2)(nlog(n) - n)$. Hence, $log(n!) = \Theta(nlog(n))$.\\
	
	\textbf{Problem 1.5:} Refer question in book.
	
	Based on the hint, we see that the harmonic series is $O(log_2(n))$ \& $\Omega(log_4(n))$. Thus, the given  harmonic series is $\Theta(log(n))$.\\
	
	\textbf{Problem 1.6:}Prove that the grade-school multiplication algorithm (Refer book for page \#), when applied to binary numbers, always gives the right answer.
	
	It works for binary numbers too. Grad-school multiplication algorithm for ($x\times y$), multiply $x$ with every bits of $y$ and each level shift multiplications results to left by one decimal places (which is similar to shifting one bit left) followed by addition.\\
	
	\textbf{Problem 1.7:}How long does the recursive multiplication algorithm (Refer book for page \#) take to multiply	an n-bit number by an m-bit number? Justify your answer.
	
	At each level, the m-bit number get reduced by one bit (being half) and you either multiply by 2 or add n-bit number with m-k bit number where k is the level. that's $O(max(m,n))$ at each step. So $O(mn)$.\\
	
	\textbf{Problem 1.8:}Justify the correctness of the recursive division algorithm given in page (Refer book for page \#), and show that it takes time $O(n^2)$ on n-bit inputs.	
	
	There are n recursive calls because of n halving that need to take place. Then, before each recursive call ends, there are arithmetic operations that are $O(n)$.
	Thus, the time complexity is $O(n^2)$. This algorithm is correct because what it is essentially doing is dividing 1 by the divisor, then doubling both the quotient and remainder, and then checking to see if the remainder has overflowed. It continues upwards until the original value is reached, at which point the quotient	and remainder will be correct.\\
	
	\textbf{Problem 1.9:}Refer question in book.
	
	Since,
	\begin{align*}
	x - x' &= Np\\
	y - y' &= Nq\\
	\Rightarrow (x+y) - (x' + y') &= (x - x') + (y - y')\\
	(x+y) - (x' + y') &= N(p+q)\\
	\Rightarrow (x - x')\times(y - y') &= N^2pq\\
	xy - xy' - x'y + x'y' &= N^2pq\\
	xy - x'y' - xy' - x'y + 2x'y' &= N6pq\\
	xy - x'y' -(x - x')y' - x'(y - y') &= N^2pq\\
	xy - x'y' &= N^2pq + Npy' + Nqx'\\
	xy - x'y' &= N(Npq + py'+ qx')
	\end{align*}
	
	\textbf{Problem 1.10:}Show that if $a \equiv b (mod N)$ and if M divides N then $a \equiv b (mod M)$.
	
	Since, $a - b = kN$ \& $N = pM$, then $a - b = kpM$ \& hence, $a \equiv b (mod M)$.\\
	
	\textbf{Problem 1.11:}Is $4^{1536} - 9^{4824}$ divisible by 35?
	
	Since, $4^6 = 1 (mod 35) \Rightarrow 4^6 - 1 = 35p$ \& $9^6 = 1 (mod 35) \Rightarrow 9^6 - 1 = 35q$. Thus, $4^6 - 9^6 = 35(p-q) \Rightarrow 4^6 - 9^6 = 0(mod 35)$ \& $4^{1536} - 9^{4824} = (4^6)^{256} - (9^6)^{804} = 0(mod 35)$.\\
	
	\textbf{Problem 1.12:}What is $2^{2^{2006}}(mod 3)$?
	
	$2 = -1 (mod 3) \Rightarrow (-1)^{2^{2006}} = 1 (mod 3)$. Hence, answer is 1.\\
	
	\textbf{Problem 1.13:}Is the difference of $5^{30,000}$ and $6^{123,456}$ a multiple of 31?
	
	Since, $5^4 = 5 (mod 31) \Rightarrow 5^4 - 5 = 31p$ \& $6^2 = 5 (mod 31) \Rightarrow 6^2 - 5 = 31q$. Thus, $5^4 - 6^2 = 31(p-q) \Rightarrow 5^4 - 6^2 = 0(mod 31)$ \& $5^{30,000} - 6^{123,456} = (5^4)^{7500} - (6^2)^{61728} = 0(mod 35)$.\\
		
	\textbf{Problem 1.14:}Suppose you want to compute the nth Fibonacci number $F_n$ , modulo an integer p. Can you find an efficient way to do this?
	
	There are $log(n)$ multiplication step and each multiplication is at most $T(log(p))$ long. Hence, $O(log(n)Tlog(p))$.\\
	
	\textbf{Problem 1.15:}Determine necessary and sufficient conditions on $x$ and $c$ so that the following holds: for any $a$, $b$, if $ax \equiv bx$ mod $c$, then $a \equiv b$ mod $c$.
	
	$ax \equiv bx$ mod $c \Rightarrow c$ must divide $(a-b)x$ and $a \equiv b$ mod $c \Rightarrow c$ must divide $(a-b)$. Hence, necessary and sufficient conditions on  $x$ and $c$ is $gcd(c,x) = 1$.\\
	
	\textbf{Problem 1.16:}The algorithm for computing $a^b$ mod $c$ by repeated squaring does not necessarily
	lead to the minimum number of multiplications. Give an example of $b > 10$	where the exponentiation can be performed using fewer multiplications, by some other method.
	
	Refer solution to problems 1.11, 1.12 \& 1.13.\\
	
	

	
\end{document}
