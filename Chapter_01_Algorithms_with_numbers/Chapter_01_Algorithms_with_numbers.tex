\documentclass[11pt]{article} 

\usepackage[utf8]{inputenc} 
\usepackage{listings}

%%% PAGE DIMENSIONS
\usepackage{geometry} 
\geometry{letterpaper} % or 
%\geometry{a4paper} % or letterpaper (US) or a5paper or....
% \geometry{margin=2in} % for example, change the margins to 2 inches all round
% \geometry{landscape} % set up the page for landscape
%   read geometry.pdf for detailed page layout information

\usepackage{graphicx} % support the \includegraphics command and options

% \usepackage[parfill]{parskip} % Activate to begin paragraphs with an empty line rather than an indent

%%% PACKAGES
\usepackage{booktabs} % for much better looking tables
\usepackage{array} % for better arrays (eg matrices) in maths
\usepackage{paralist} % very flexible & customisable lists (eg. enumerate/itemize, etc.)
\usepackage{verbatim} % adds environment for commenting out blocks of text & for better verbatim
\usepackage{subfig} % make it possible to include more than one captioned figure/table in a single float
% These packages are all incorporated in the memoir class to one degree or another...
\usepackage{amsmath} %for mathematical equations
%%% HEADERS & FOOTERS
\usepackage{fancyhdr} % This should be set AFTER setting up the page geometry
\usepackage{enumitem}
\pagestyle{fancy} % options: empty , plain , fancy
\renewcommand{\headrulewidth}{0pt} % customise the layout...
\lhead{}\chead{}\rhead{}
\lfoot{}\cfoot{\thepage}\rfoot{}

%%% SECTION TITLE APPEARANCE
\usepackage{sectsty}
\allsectionsfont{\sffamily\mdseries\upshape} % (See the fntguide.pdf for font help)
% (This matches ConTeXt defaults)

%%% ToC (table of contents) APPEARANCE
\usepackage[nottoc,notlof,notlot]{tocbibind} % Put the bibliography in the ToC
\usepackage[titles,subfigure]{tocloft} % Alter the style of the Table of Contents
\renewcommand{\cftsecfont}{\rmfamily\mdseries\upshape}
\renewcommand{\cftsecpagefont}{\rmfamily\mdseries\upshape} % No bold!

%%% END Article customizations

%%% The "real" document content comes below...

\title{Algorithms\\By\\Sanjoy Dasgupta Christos Papadimitriou Umesh Vazirani \\ Chapter 1: Algorithms with numbers}
\author{Yukteshwar Baranwal}
%\date{} % Activate to display a given date or no date (if empty),
% otherwise the current date is printed 
\newcommand{\dd}[1]{\mathrm{d}#1}

\begin{document}
	\maketitle
	
	\textbf{Problem 1.1:} Show that in any base $b \geq 2$, the sum of any three single-digit numbers is at
	most two digits long.
	
	Maximum possible number in base $b$ is $b-1$ e.g. 1 in binary($b = 2$) and 9 in decimal($b=10$). On adding $b-1$ thrice, the result will be $3\times (b-1)$, which is at most 2 bit long as $3\times (b-1) < b^3$.\\
	
	\textbf{Problem 1.2:} Show that any binary integer is at most four times as long as the corresponding decimal integer. For very large numbers, what is the ratio of these two lengths,	approximately?
	
	A number of $n$ in base $b$ has length $log(n)/log(b)$. So , length for binary is $log(n)/log(2)$ \& for decimal is $log(n)/log(10)$. Their ratio is given by $log(10)/log(2) = 1/0.3010 > 3$ and its ceil value is 4.\\
	
	\textbf{Problem 1.3:} A d-ary tree is a rooted tree in which each node has at most d children. Show that any d-ary tree with n nodes must have a depth of $\Omega(log(n)/log(d))$. Can you give a precise formula for the minimum depth it could possibly have?
	
	For n child nodes and for a complete tree, minimum depth would be $log(n)/log(d)$. The sum of nodes over levels is a geometric series sum having value $(d^{k+1}-1)/(d-1)$ where k is depth of tree and this sum would be n. Hence, k would be $\Omega(log(n)/log(d))$. Precise representation would be $\lfloor log_d(n)\rfloor$.\\
	
	\textbf{Problem 1.4:} Show that $log(n!) = \Theta(nlog(n))$.
	
	Using hint, $n! < n^n$ \& $n! > (n/2)^{n/2}$. Hence, $log(n!) < nlog(n)$ \& $log(n!) < (n/2)log(n/2)$. The lower bound can also be expressed as $log(n!) < (1/2)(nlog(n) - n)$. Hence, $log(n!) = \Theta(nlog(n))$.
	
	
\end{document}
