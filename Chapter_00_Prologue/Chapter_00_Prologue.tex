\documentclass[11pt]{article} 

\usepackage[utf8]{inputenc} 
\usepackage{listings}

%%% PAGE DIMENSIONS
\usepackage{geometry} 
\geometry{letterpaper} % or 
%\geometry{a4paper} % or letterpaper (US) or a5paper or....
% \geometry{margin=2in} % for example, change the margins to 2 inches all round
% \geometry{landscape} % set up the page for landscape
%   read geometry.pdf for detailed page layout information

\usepackage{graphicx} % support the \includegraphics command and options

% \usepackage[parfill]{parskip} % Activate to begin paragraphs with an empty line rather than an indent

%%% PACKAGES
\usepackage{booktabs} % for much better looking tables
\usepackage{array} % for better arrays (eg matrices) in maths
\usepackage{paralist} % very flexible & customisable lists (eg. enumerate/itemize, etc.)
\usepackage{verbatim} % adds environment for commenting out blocks of text & for better verbatim
\usepackage{subfig} % make it possible to include more than one captioned figure/table in a single float
% These packages are all incorporated in the memoir class to one degree or another...
\usepackage{amsmath} %for mathematical equations
%%% HEADERS & FOOTERS
\usepackage{fancyhdr} % This should be set AFTER setting up the page geometry
\usepackage{enumitem}
\pagestyle{fancy} % options: empty , plain , fancy
\renewcommand{\headrulewidth}{0pt} % customise the layout...
\lhead{}\chead{}\rhead{}
\lfoot{}\cfoot{\thepage}\rfoot{}

%%% SECTION TITLE APPEARANCE
\usepackage{sectsty}
\allsectionsfont{\sffamily\mdseries\upshape} % (See the fntguide.pdf for font help)
% (This matches ConTeXt defaults)

%%% ToC (table of contents) APPEARANCE
\usepackage[nottoc,notlof,notlot]{tocbibind} % Put the bibliography in the ToC
\usepackage[titles,subfigure]{tocloft} % Alter the style of the Table of Contents
\renewcommand{\cftsecfont}{\rmfamily\mdseries\upshape}
\renewcommand{\cftsecpagefont}{\rmfamily\mdseries\upshape} % No bold!

%%% END Article customizations

%%% The "real" document content comes below...

\title{Algorithms\\By\\Sanjoy Dasgupta Christos Papadimitriou Umesh Vazirani \\ Chapter 0: Prologue}
\author{Yukteshwar Baranwal}
%\date{} % Activate to display a given date or no date (if empty),
% otherwise the current date is printed 
\newcommand{\dd}[1]{\mathrm{d}#1}

\begin{document}
	\maketitle
	
	\textbf{Problem 0.1:} In each of the following situations, indicate whether $f = O(g)$, or $f = \Omega(g)$, or both (in which case $f = \Theta(g)$).
	
	\textbf{Understanding:} 
	\begin{enumerate}
		\item $f = O(g)$ means that as n ( the problem size tends to infinity) $f$ is bounded above by $g$.
		\item $f = \Omega(g)$ means that as n ( the problem size tends to infinity) $f$ is bounded below by $g$.
		\item $f = \Theta(g)$ means that as n ( the problem size tends to infinity) $f$ is bounded above and below by $g$.
	\end{enumerate}
	
	
	\begin{enumerate}[label=(\alph*)]
		\item $f(n) = n - 100$, $g(n) = n - 200$ $\Rightarrow f = \Theta(g)$.
		
		\item $f(n) = n^{1/2}$, $g(n) = n^{2/3}$ $\Rightarrow f = O(g)$.
		
		\item $f(n) = 100n + log(n)$, $g(n) = n + (log(n))^2$ $\Rightarrow f = \Theta(g)$.
		
		Since, $(log(n))^2 = O(n)$
		
		\item $f(n) = nlog(n)$, $g(n) = 10nlog(10n)$ $\Rightarrow f = \Theta(g)$.
		
		\item $f(n) = log(2n)$, $g(n) = log(3n)$ $\Rightarrow f = \Theta(g)$.
		
		\item $f(n) = 10log(n)$, $g(n) = log(n^2)$ $\Rightarrow f = \Theta(g)$.
		
		\item $f(n) = n^{1.01}$, $g(n) = n(log(n))^2$ $\Rightarrow f = \Omega(g)$.
		
		\item $f(n) = n^2/log(n)$, $g(n) = n(log(n))^2$ $\Rightarrow f = \Omega(g)$.
		
		\item $f(n) = n^{0.1}$, $g(n) = (log(n))^{10}$ $\Rightarrow f = \Omega(g)$.
		
		\item $f(n) = (log(n))^{log(n)}$, $g(n) = n/(log(n)$ $\Rightarrow f = \Omega(g)$.
		
		\item $f(n) = \sqrt{n}$, $g(n) = (log(n))^3$ $\Rightarrow f = \Omega(g)$.
		
		\item $f(n) = n^{1/2}$, $g(n) = 5^{log_2(n)}$ $\Rightarrow f = O(g)$.
		
		Since, $5^{log_2(n)} = n^{log_2(5)} = n^{2.32} > n^{1/2}$
		
		\item $f(n) = n2^n$, $g(n) = 3^n$ $\Rightarrow f = \Theta(g)$.
		
		Since, $$log(f(n)) = n + log(n)$$
		$$log(g(n)) = nlog(3)$$ 
		$$\Rightarrow log(f(n)) = \Theta(log(g(n)))$$
		$$\Rightarrow f(n) = \Theta(g(n))$$
		
		\item $f(n) = 2^n$, $g(n) = 2^{n+1}$ $\Rightarrow f = \Theta(g)$.
		
		Since, $2^{n+1} = 2\times 2^n$.
		
		\item $f(n) = n!$, $g(n) = 2^n$ $\Rightarrow f = \Omega(g)$.
		
		Since, $ n^n > n! > (n/2)^{n/2}$
		
		\item $f(n) = (log(n))^{log(n)}$, $g(n) = 2^{(log_2n)^2}$ $\Rightarrow f = O(g)$.
		Since,
		\begin{align*}
		f(n) &= (log(n))^{log(n)}\\
		&= (2^{log(log(n))})^{log(n)}\\
		&= (2^{log(n)})^{log(log(n))}\\
		&= n^{log(log(n))}\\
		g(n) &= 2^{log(n)log(n)}\\
		&= n^{log(n)}
		\end{align*}
		
		\item $f(n) = \sum_{i=1}^ni^k$, $g(n) = n^{k+1}$ $\Rightarrow f = O(g)$.
		
		Since, $f(n) = \sum_{i=1}^ni^k < n\times n^k$
		
	\end{enumerate}
	
	\textbf{Problem 0.2:}Show that, if $c$ is a positive real number, then $g(n) = 1 + c + c^2 + ... + c^n$ is:
	
	\begin{enumerate}[label=(\alph*)]
		\item $\Theta(1)$ if $c < 1$\\
		If $c<1$, then $g(n)$ can be expressed as $\frac{1-c^{n+1}}{1-c}$. For higher values of $n$, it reaches to limiting value i.e., $\frac{1}{1-c}$. I can always get a constant 
		number greater than this and lower bound is 1 thus it is $\Theta(1)$. 
		
		\item $\Theta(n)$ if $c = 1$\\
		If $c<1$, then $g(n) = n = \Theta(n)$
		
		\item $\Theta(c^n)$ if $c > 1$\\
		If $c>1$, then $g(n)$ can be expressed as $\frac{c^{n+1}-1}{c-1}$. It is bounded from both sides by $c^n$ functions and thus $\Theta(c^n)$.
	\end{enumerate}
	
	\textbf{Problem 0.3:}The Fibonacci numbers $F_0$, $F_1$, $F_2$, ..., are defined by rule: $F_0 = 0$, $F_1 = 1$, $F_n = F_{n-1} + F_{n-2}$. In this problem we will confirm that this sequence grows exponentially fast and obtain some bounds on its growth.
	
	\begin{enumerate}[label=(\alph*)]
		\item Use induction to prove that $F_n \geq 2^{0.5n}$ for $n \geq 6$.\\
		Base case: $F_6 = F_5 + F_4 = 5 + 3 = 8 \geq 2^{0.5\times 6}$.
		
		Similarily, $F_7 = F_6 + F_5 = 8 + 5 = 13 \geq 2^{0.5\times 7}$.
		
		Inductive hypothesis: For an index $k > 6$, if $F_k \geq 2^{0.5k}$, then $F_{k+1} \geq 2^{0.5(k+1)}$.
		
		Inductive step: $F_{k+1} = F_k + F_{k-1} \geq 2^{0.5k} + 2^{0.5(k-1)} = 2^{0.5k}\left(\frac{\sqrt{2} + 1}{\sqrt{2}}\right) = 2^{0.5k}\sqrt{2}\left(\frac{\sqrt{2} + 1}{2}\right) = 2^{0.5(k+1)}\left(\frac{\sqrt{2} + 1}{2}\right) \geq 2^{0.5(k+1)}$
		
		\item Find a constant $c < 1$ such that $F_n \leq 2^{cn}$ for all $n \geq 0$. Show that your answer
		is correct.
		
		\begin{align*}
		F_n &= F_{n-1} + F_{n-2} \leq 2^{cn}\\
		\Rightarrow & 2^{c(n-1)} + 2^{c(n-2)} \leq 2^{cn}\\
		\Rightarrow & 2^{-c} + 2^{-2c} \leq 1\\
		\Rightarrow & x^2 + x -1 \leq 0
		\end{align*}
		where $x = 2^{-c} > 0$. Solving for $x$.
		\begin{align*}
		& 0 \leq x \leq \frac{\sqrt{5}-1}{2}\\
		\Rightarrow & 0 \leq 2^{-c} \leq \frac{\sqrt{5}-1}{2}\\
		\Rightarrow & -\infty \leq -c \leq log(\sqrt{5}-1) - 1\\
		\Rightarrow & c \geq log(\sqrt{5}-1) - 1
		\end{align*}
		Since, $c < 1$, hence $log(\sqrt{5}-1) - 1 \leq c < 1$.
		
		\item What is the largest c you can find for which $F_n = \Omega(2^{cn})$?\\
		The minimum value of $c$ in last problem would satisfy $F_n = \Omega(2^{cn})$. Hence, max value of $c$ is $log(\sqrt{5}-1) - 1$.		
	\end{enumerate}
	
	\textbf{Problem 0.4:}Is there a faster way to compute the nth Fibonacci number than by \textit{fib2}? Refer question in book.
	
	\begin{enumerate}[label=(\alph*)]
		\item Show that two $2\times 2$ matrices can be multiplied using 4 additions and 8 multiplications.\\		
		$$\begin{bmatrix}
			a & b\\c & d
		\end{bmatrix} \times
		\begin{bmatrix}
		e & f\\g & h
		\end{bmatrix} = 
		\begin{bmatrix}
			ae+bg & ce+dg\\af+bh & cf+dh
		\end{bmatrix}$$
		Hence, there are 4 additions and 8 multiplications.
		
		\item Refer problem in book.\\
		Computing $X^n$ is multiplication by squaring. Recursively, if you have $X^n$, this is $(X^{n/2})^2$ if N is even or $X\times X^{n-1}$ if $n$ is odd. Then apply this recursively. This alternates between halving and reducing the problem size by one at each iteration. So it will take $log(n)$ times. 
		
		\item Refer problem in book.\\
		Intermediate results are multiplication and addition of two numbers. We start with just one bit. When we multiply two 1 bit numbers and add them, we get at most 2 bits numbers. The general idea is by adding two numbers that have n-1 
		bits, we get at most a n bit number.
		
		\item Refer problem in book.\\
		There are $log(n)$ iterations and each iteration has 8 multiplications with running time $M(n)$. Hence, the running time of algorithm is $O(M(n)log(n))$.
		
		\item Refer problem in book.\\
		We can prove this using induction. Our base cases are $F_0$ and $F_1$, which clearly require $O(M(n))$ time. Now, assume that the claim holds for $1 \leq n < k$. If $k$ is odd, then $F_k = F\times F_{\lfloor k/2 \rfloor}\times F_{\lfloor k/2 \rfloor}$. By the inductive hypothesis, determining $F_{\lfloor k/2 \rfloor}$ requires $O(M(k/2)) = O(M(k))$ time. Multiplying these arrays requires $O(M(k))$ time as well, so $F_k$ can be found in $O(M(k))$ time. Similarly, $F_k$ can be computed in $O(M(k))$ time if $k$ is even. Therefore, the claim holds by induction. 
	\end{enumerate}

\end{document}
